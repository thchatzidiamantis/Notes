\documentclass{scrartcl}

\usepackage{amsmath,amsthm,latexsym}
\usepackage{amssymb}
\usepackage[mathscr]{euscript}
\usepackage{mathtools}
\usepackage[bitstream-charter]{mathdesign} 
\usepackage[T1]{fontenc}

\usepackage[all,arc]{xy}
\usepackage{tikz}
\usepackage{tikz-cd}

\usepackage[textwidth=2.65cm,textsize=scriptsize,color=orange!40,linecolor=green!40!black,colorinlistoftodos]{todonotes} 

\usepackage[hidelinks]{hyperref}
\hypersetup{colorlinks=true, linkcolor=teal, urlcolor=teal, citecolor=purple}

\newcommand{\textbi}[1]{\textbf{\textit{#1}}}

\newcommand{\C}{\mathscr{C}}
\newcommand{\D}{\mathscr{D}}
\newcommand{\F}{\mathscr{F}}
\newcommand{\N}{\mathscr{N}}
\newcommand{\bN}{\mathbb{N}}
\newcommand{\bZ}{\mathbb{Z}}
\newcommand{\bR}{\mathbb{R}}
\newcommand{\bfO}{\mathbf{O}}
\newcommand{\cS}{\mathscr{S}}
\newcommand{\X}{\mathscr{X}}

\newcommand{\Top}{\mathscr{T}\mathrm{op}}
\newcommand{\id}{\mathrm{id}}
\newcommand{\Obj}{\mathrm{Obj}}
\newcommand{\Ho}{\mathrm{Ho}}
\newcommand{\Hom}{\mathrm{Hom}}
\newcommand{\const}{\mathrm{const}}
\newcommand{\Set}{\cS\mathrm{et}}
\newcommand{\sS}{s\cS}
\newcommand{\Fun}{\mathrm{Fun}}
\newcommand{\Aut}{\mathrm{Aut}}
\newcommand{\Map}{\mathrm{Map}}
\newcommand{\Nat}{\mathrm{Nat}}
\newcommand{\Sq}{\mathrm{Sq}}
\newcommand{\colim}{\mathrm{colim}}

\swapnumbers
\newtheorem{defin}[subsection]{Definition}
\newtheorem{prop}[subsection]{Proposition}
\newtheorem{cor}[subsection]{Corollary}
\newtheorem{lem}[subsection]{Lemma}
\newtheorem{thm}[subsection]{Theorem}
\newtheorem{rem}[subsection]{Remark}
\newtheorem{ex}[subsection]{Example}
\newtheorem{cons}[subsection]{Construction}

\title{Talk 3: Equivariant Orthogonal Spectra}
\author{Theofanis Chatzidiamantis-Christoforidis}
\date{April 2024}

\begin{document}

\renewcommand{\abstractname}{\vspace{-\baselineskip}}

\maketitle
\begin{abstract}
   These notes are mostly a (condensed) combination of \cite[3.1]{GHT} and \cite[1-3]{Sch23}. If you spot any mistakes or typos, please contact me at \href{mailto:s92tchat@uni-bonn.de}{s92tchat@uni-bonn.de}. 
\end{abstract}

\section{Preliminaries}

\begin{defin}
    The $n$-th \textbi{orthogonal group} $O(n)$ is the group of isometries of $\bR^n$ as a euclidean vector space (i.e., preserving the origin). As a matrix group, $$O(n)=\{A\in GL_n(\bR)\ |\ A^TA=AA^T=I_n\}$$ is the subgroup of $GL_n(\bR)$ of all orthogonal matrices. 
    \par Generalizing the first presentation, we similarly write $O(V)$ for the orthogonal group of an arbitrary real finite-dimensional inner product space $V$.
\end{defin}

Recall that a \textbi{representation} of a group $G$ (or $G$-representation) on a $K$-vector space $V$ is a group homomorphism $G\to GL(V)$. If $V$ is $n$-dimensional for some finite $n$, we can choose a basis and identify $GL(V)$ with $GL_n(K)$.
\par We can also view a representation as a \textit{linear} group action $G\times V\to V$: For a homomorphism $\rho:G\to GL(V)$, define the associated action to be $(g, v)\mapsto \rho(g)v$. Conversely, for a linear group action $a:G\times V\to V$, use the representation sending $g$ to the matrix associated to the linear map $a(g)$.

\begin{rem}
    In common notation, we sometimes identify the vector space $V$ with the representation.
\end{rem}

\begin{defin}
    An \textbi{orthogonal representation} of a group $G$ is a homomorphism $G\to O(V)$ for an inner product space $V$. 
\end{defin} 

\begin{defin}
    We denote by $RO(G)$ the group completion of the monoid of isomorphism classes of orthogonal $G$-representations.
\end{defin}

\begin{rem}
    Every orthogonal representation is just a representation with a restriction on its image. If $V$ is a finite dimensional real inner product space, we can reformulate the definition: An orthogonal $G$-representation is an action of $G$ on $V$ by linear isometries.
\end{rem}

\textbf{Note:} For the rest of the talk, we assume $G$-representations are orthogonal unless stated otherwise.

\begin{defin}
    Let $G$ be a finite group. The \textbi{regular representation} $\rho_G$ of $G$ is the free vector space $\bR[G]$ together with the action
    \begin{align*}
        G\times \bR[G]&\to \bR[G] \\
        (g,\small \sum_ir_ih_i)&\mapsto \small \sum_ir_i(gh_i)
    \end{align*}
\end{defin}

\begin{defin}
\begin{itemize}
    \item[] 
    \item Let $V$ be a finite-dimensional real $G$-representation. We call the one-point compactification of $V$ its \textbi{representation sphere}, and we regard it as a based $G$-space with the basepoint at infinity.
    \item If $V$ is endowned with a scalar product, we denote by $S(V)$ its unit sphere.
\end{itemize}    
\end{defin} 

Note that for $\bR^n$ with a trivial action, the representation sphere is $S^n$ and the unit sphere is $S^{n-1}$.

\begin{prop}
    For any finite group $G$ (actually, for any compact Lie group) and $V$ be a finite-dimensional real orthogonal $G$-representation. Then the representation sphere $S^V$ admits a $G$-$CW$ structure.
\end{prop} 

%\begin{defin}
    %The \textbi{orbit category} $\mathcal{O}_G$ of a finite group $G$ is the category with objects the $G$-sets $G/H$ for %all subgroups $H\leq G$, and $G$-equivariant maps as morphisms.
%\end{defin}


\subsection*{Spaces and the Indexing Category $\mathbf{O}$}

We will be working with \textit{compactly generated weakly Hausdorff} spaces, which we now simplify to just \textit{spaces}. We denote the category of such spaces by $\mathsf{T}$ and the category of based spaces by $\mathsf{T}_*$. Similarly, we have the categories $G\mathsf{T}$ and $G\mathsf{T}_*$ of $G$-spaces and based $G$-spaces, with morphisms being (based) $G$-equivariant maps.
\begin{rem}
    Recall that the category $\mathsf{T}$ has small (co)limits.
\end{rem} 

Next, we have to think about $G$-representations. When defining, for example, stable homotopy groups of $G$-spectra, we will want to work over every finite-dimensional orthogonal representation, at least up to some notion of isomorphism. We introduce the notion of a universe to make this rigorous.

\begin{defin}{\normalfont \cite[1.1.12]{GHT}}
    Let $G$ be a compact Lie group. A $G$-\textbi{universe} is an orthogonal representation $\mathcal{U}$ of countably infinite dimension such that
    \begin{enumerate}
        \item $\mathcal{U}$ has nontrivial fixed points.
        \item If a finite-dimensional $G$-representation $V$ embeds into $\mathcal{U}$, then a countably infinite direct sum of copies of $V$ also embeds into $\mathcal{U}$.
    \end{enumerate}
    $\mathcal{U}$ is called \textbi{complete} if every finite-dimensional $G$-representation embeds into it.
\end{defin}

\begin{rem}
    Two important facts are relevant to us: 
    \begin{enumerate}
        \item Complete $G$-universes always exist.
        \item If $H\leq G$ is closed and $\mathcal{U}$ is a complete $G$-universe, then $\mathcal{U}$ as an $H$-representation is a complete $H$-universe. 
    \end{enumerate}
\end{rem}

We fix a complete $G$-universe $\mathcal{U}_G$ and write $s(\mathcal{U}_G)$ for the \textit{poset of finite-dimensional $G$-subrepresentations} in $\mathcal{U}_G$, ordered by inclusion. 

\begin{defin}\label{emb}
    Let $V,W$ be finite-dimensional inner product spaces. 
    \begin{itemize}
        \item Define $L(V,W)$ to be the set of linear isometric embeddings $V\hookrightarrow W$. 
        \item For $\varphi\in L(V,W)$, we write $\varphi^\perp$ for the orthogonal complement $W- \varphi(V)$.
        \item We equip $L(V,W)$ with the topology induced by the bijection 
              \begin{align*}
                  O(W)/O(\varphi^\bot) &\to L(V,W) \\
                  A\cdot O(\varphi^\bot) &\mapsto A\circ \varphi
              \end{align*}
    \end{itemize} 
\end{defin} 

Alternatively, the topology on $L(V,W)$ is the \textbi{Stiefel manifold} topology: Recall that the $k$-th Stiefel manifold in $\bR^n$ is the set of $k$-tuples of orthonormal vectors in $\bR^n$, with the subspace topology from $\bR^{nk}$. In particular, this is a \textit{compact} space.  

\begin{rem} \label{isom}
    For $V=W$, $L(V,V)$ just consists of the linear isometries, so $L(V,V)=O(V)$.
\end{rem}

\begin{cons}
    We define the category $\mathbf{O}$ with
    \begin{itemize}
        \item[Objects:] all finite-dimensional real inner product spaces.
        \item[Morphisms:] Let $V,W$ be two objects of $\bfO$. Define $$\Hom_{\bfO}(V,W)= \bfO(V,W)\coloneqq\text{Th}(\xi(V,W))$$ i.e., the Thom space of the vector bundle 
        \[\begin{tikzcd}
            \xi(V,W)\coloneqq \{(w,\varphi)\in W\times L(V,W)\ |\ w\perp\varphi(V)\} \arrow[d, "pr_2"] \\
            L(V,W)
        \end{tikzcd}\]
        Note that in \ref{emb} we made sure that the base space of this bundle is compact. In this case, the Thom space is just the one-point compactification of the total space.
        \item[Composition:] We have the operation 
        \begin{align*}
            \circ: \bfO(V,W)\wedge \bfO(U,V)&\to \bfO(U,W) \\
            (w,\varphi)\wedge (v,\psi) &\mapsto (w+\varphi(v), \psi\circ\varphi)
        \end{align*} induced by the bundle map $\xi(V,W)\wedge \xi(U,V)\to \xi(U,W)$ covering the usual composition map for $L$.
        \par We have associativity of composition and identities given by $(0, id_V)\in \bfO(V,V)$.
    \end{itemize} 
\end{cons}

\begin{rem}
    This category is topologically enriched, meaning that the hom-sets $\bfO(V,W)$ have a topological space structure as Thom spaces, and the composition operation is continuous. We will immediately put this to use. 
\end{rem}

\section{Equivariant Orthogonal Spectra}

\begin{defin}\label{def}
    Fix a finite group (more generally, a compact Lie group) $G$.
    \begin{itemize}
        \item An \textbi{orthogonal spectrum} is a based continuous functor $\bfO\to \mathsf{T}_*$.
        \item An \textbi{orthogonal $G$-spectrum} is a based continuous functor $\bfO\to G\mathsf{T}_*$. 
    \end{itemize}
    A morphism of orthogonal ($G$-)spectra is a natural transformation of functors. We denote the category of orthogonal spectra (resp. orthogonal $G$-spectra) by $\mathsf{Sp}$ (resp. $G\mathsf{Sp}$).
\end{defin}

As in sequential spectra, we want a way to move between values at different inner product spaces after smashing with spheres. Here we use the topological enrichment of $\bfO$ to be able to apply $X$ after smashing with a hom-space of $\bfO$. 

\begin{cons}[Structure maps]
    Let $V,W$ be (finite-dimensional, real) inner product spaces and $X$ an orthogonal ($G$-)spectrum. We define a continuous based map 
    \begin{align*}
        i_V:S^V&\to \bfO(W, V\oplus W) \\
         x&\mapsto ((x,0), \text{incl}_W)
    \end{align*}
    Using this, we define the \textbi{structure map} $\sigma_{V,W}:S^V\wedge X(W)\to X(V\oplus W)$ to be the composite 
    $$S^V\wedge X(W)\xrightarrow{i_V\wedge X(W)} \bfO(W, V\oplus W)\wedge X(W)\xrightarrow{X} X(V\oplus W)$$
    and the \textbi{opposite structure map} $\sigma_{V,W}^{op}:X(V)\wedge S^W\to X(V\oplus W)$ as
    $$X(V)\wedge S^W\xrightarrow{\text{twist}}S^W\wedge X(V)\xrightarrow{\sigma_{W,V}}X(W\oplus V)\xrightarrow{X((w,v)\mapsto (v,w))}X(V\oplus W)$$
    In the case of $G$-spectra, these maps are equivariant with $G$ also acting on the representation spheres. 
\end{cons}

\subsection*{Getting An Explicit Description}

Next, we want to take this abstract definition and produce an equivalent description that relies on the usual notion of sequential spectra at its core, with extra structure. We give an idea of how one can reach such a definition but do not translate all the coherence conditions. \\ 

Let us begin with orthogonal spectra. We give the alternative definition:

\begin{defin}
    An orthogonal spectrum consists of the following data:
    \begin{itemize}
        \item A sequence of pointed spaces $X_n$ for all $n\in\bN$.
        \item A basepoint-prerserving left $O(n)$-action on every $X_n$.
        \item Based structure maps $\sigma_n:X_n\wedge S^1\to X_{n+1}$.
    \end{itemize}
    This is subject to the condition that the iterated structure maps $\sigma^m:X_n\wedge S^m\to X_{n+m}$ are $O(n)\times O(m)$-equivariant, where $O(n)\times O(m)$ acts on $X_{n+m}$ by restricting the action of $O(n+m)$ along orthogonal sum. \par
    A morphism of orthogonal spectra $f:X\to Y$ consists of a sequence of $O(n)$-equivariant based maps $f_n:X_n\to Y_n$ that are compatible with the structure maps.
\end{defin}

To see that this determines an orthogonal spectrum in the sense of \ref{def}, first note that a finite-dimensional real inner product space is always isomorphic to some $\bR^n$ with the standard inner product. Thus, an orthogonal spectrum is determined by its values on $\bR^n$ for $n\in\bN$, i.e., the $X_n\coloneqq X(\bR^n)$.
\par Going the other way, we explain where these actions are coming from: Looking at the mapping spaces in $\bfO$, recall that by \ref{isom}, we know that $L(\bR^n,\bR^n)=O(n)$. As isometries are bijections, $$\xi(\bR^n,\bR^n)=\{0\}\times L(\bR^n,\bR^n)\cong O(n)$$ Taking one-point compactifications and putting everything together results in a homeomorphism 
\begin{align*}
    O(n)_+&\xrightarrow{\cong} \bfO(\bR^n,\bR^n) \\
    A&\mapsto (0,A)
\end{align*}
Thus this hom-space encodes the based $O(n)$-action on $X_n$. The map $\sigma_n$ corresponds to $\sigma_{\bR, \bR^n}$ as constructed earlier. \\

We now move to orthogonal $G$-spectra. By definition of $G\mathsf{T}_*$ and functoriality, we are really adding a continuous $G$-action together with rules that everything is now $G$-equivariant.

\begin{defin}\label{alt}
    An orthogonal $G$-spectrum is an orthogonal spectrum equipped with a $G$-action through automorphisms of orthogonal spectra.\par Equivalently, it consists of the following data:
    \begin{itemize}
        \item A sequence of pointed spaces $X_n$ for all $n\in\bN$.
        \item A basepoint-prerserving left $O(n)\times G$-action on every $X_n$.
        \item $G$-equivariant based structure maps $\sigma_n:X_n\wedge S^1\to X_{n+1}$, with respect to the action given by the $G$-action on $X_n$ and $X_{n+1}$ and the trivial action on the sphere. 
    \end{itemize}
    This is subject to the condition that the iterated structure maps $\sigma^m:X_n\wedge S^m\to X_{n+m}$ are $O(n)\times O(m)$-equivariant.
\end{defin}

\begin{rem}
    The condition above implies that the maps $\sigma^m$ are also equivariant with respect to the $G$-actions on $X_n$ and $X_{n+m}$ and the trivial $G$-action on $S^m$.
\end{rem}

\subsection*{On $G$-representations}

So what happens when we have a $G$-representation on an inner product space $V$? This means that $G$ acts on both $V$ and $X$, and we want to produce the evaluation of $X$ on $V$ as well.

\begin{cons}
    For an $n$-dimensional $G$-representation $V$, define $$X(V)=\bfO(\bR^n, V)\wedge_{O(n)}X_n$$ i.e., the coequalizer of the $O(n)$-actions on $\bfO(\bR^n, V)$ by precomposition and on $X_n$. As a space, $X(V)$ still only depends on the dimension of $V$: Choosing an isometry $\varphi: \bR^n\to V$, the map $[\varphi, -]:X_n\to X(V)$ is a homeomorphism. We now view $X(V)$ as a $G$-space by considering the diagonal action $$g[\varphi,x]=[g\varphi,gx]$$
\end{cons}

This means that the action on $X(V)$ depdends on the action on $V$, which we will see in practice when presenting examples.

\begin{rem}
    There are other definitions of orthogonal $G$-spectra in the literature (\cite{MM}, \cite{HHR}) that use all $G$-representations or depend on the chosen $G$-universe. However, the categories defined are equivalent, see for example \cite[2.7]{Sch23}
\end{rem}

\subsection*{Examples}
We now use this second description to provide examples, as we can work similarly to the case of ``non-equivariant, non-orthogonal'' spectra we already know.

\begin{ex}[Non-equivariant sphere spectrum]
    We define the orthogonal sphere spectrum $\mathbb{S}$ as $\mathbb{S}_n\coloneqq S^n$, with the $O(n)$-action coming from the $O(n)$-action on $\bR^n$. The structure maps are given by the canonical homeomorphism $S^n\wedge S^1\to S^{n+1}$.  
\end{ex}

\begin{rem}
    This carries the extra structure of a ring spectrum, and is, in fact, the initial orthogonal ring spectrum.
\end{rem}

\begin{ex}[Orthogonal $G$-spectra from non-equivariant orthogonal spectra]\label{triv}
    Let $X$ be an equivariant orthogonal spectrum. We can turn this into a $G$-spectrum by letting $G$ act trivially.
\end{ex}

\begin{rem}
    We described how an orthogonal $G$-spectrum can take values on all $G$-representations. One may assume that the trivial action added in \ref{triv} stays trivial for all representations, but this is not true! For example, if $O(n)$ acts non-trivially, e.g. as in the sphere spectrum, then take a non-trivial orthogonal $G$-representation $G\to O(n)$. The action on $X_n$ is given by $G\times X_n\to O(n)\times X_n\to X_n$.
\end{rem}

\begin{ex}[Suspension spectra]
    Let us look at something with a non-trivial $G$-action from the start: Generalizing the case of the sphere, if we have a based $G$-space $A$, we can define the suspension spectrum $\Sigma^\infty A$ by setting $$\Sigma^\infty A(V)\coloneqq S^V\wedge A$$
    The $O(V)$-action come from the $O(V)$-action on $S^V$ and we also have a $G$-action induced from the $G$-action on $A$. The structure maps $\sigma_{U,V}$ are induced by the canonical homeomorphisms $S^U\wedge S^V\xrightarrow{\cong} S^{U\oplus V}$.
\end{ex}

We now want to outline the construction of Eilenberg-Mac Lane spectra. We need a construction of EM-spaces that inherits a ``good'' $O(n)$-action from spheres and a $G$-action from a given abelian group, thus we outline a way to define them using linearization. We point to \cite[6.4]{AGP} for details and \href{https://ncatlab.org/nlab/show/Eilenberg-Mac+Lane+space#via_linearization_of_spheres}{nLab} for a quick overview.

\begin{defin}
    Let $M$ be an abelian group and $n\in \bN$. The \textbi{reduced $M$-linearization} $M[S^n]_*$ of $S^n$ is the quotient $$M[S^n]_*\coloneqq \coprod_{k\in\bN}M^k\times (S^n)^k\big/\sim$$
    seen as finite formal sums $\sum m_ix_i$, under the relation that any multiple of the basepoint of $S^n$ is identified with $0_G$. 
\end{defin}

\begin{prop}
    Let $M$ be a countable abelian group. then $M[S^n]_*$ is a $K(M,n)$.
\end{prop}

\begin{ex}[Eilenberg-Mac Lane spectra]
    Let $M$ be a (countable) abelian group with an additive $G$-action. We define the Eilenberg-Mac Lane spectrum $HM$ by $(HM)_n\coloneqq M[S^n]_*$, with the $O(n)$-action coming from the $O(n)$-action on $S^n$ and the $G$-action coming from the $G$-action on $M$. We define the structure maps
    \begin{align*}
        \sigma_n: M[S^n]_*\wedge S^1&\to M[S^{n+1}]_* \\
        \sum_i m_ix_i \wedge y &\mapsto \sum_i m_i(x_i\wedge y)
    \end{align*}
    This spectrum has more important properties. Namely, the spaces $M[S^n]_*$ are equivariant Eilenberg-Mac Lane spaces for the coefficient system 
    \begin{align*}
        \mathcal{O}_G^{op}&\to \mathsf{Ab} \\
        G/H&\mapsto M^H
    \end{align*}
    and that EM-spectra induce a symmetric monoidal structure. This has explained nothing, so see \cite[2.13]{Sch23} for details.
\end{ex}

\subsection*{More structure}

\begin{cons}[Smash product-mapping space adjunction for orthogonal $G$-spectra]
    Let $A$ be pointed $G$-space and $X,Y$ be orthogonal $G$-spectra. We can define the orthogonal $G$-spectra:
    \begin{itemize}
        \item $X\wedge A$, by setting $(X\wedge A)(V)\coloneqq X(V)\wedge A$, with diagonal $G$-action, $O(V)$-action via its action on $X(V)$ and structure maps $$S^V\wedge X(W)\wedge A\xrightarrow{\sigma_{V,W}\wedge id_A} X(V\oplus W)\wedge A$$
        \item $\text{map}_*(A,X)$, by setting $\text{map}_*(A,X)(V)\coloneqq \text{map}_*(A,X(V))$, with $G$-action by conjugation, $O(V)$-action via its action on $X(V)$ and structure maps 
        \begin{align*}
            S^V\wedge \text{map}_*(A, X(W))&\to \text{map}_*(A, S^V\wedge X(W))\xrightarrow{\text{map}_*(A, \sigma_{V,W})} \text{map}_*(A, X(V\oplus W)) \\
            z\wedge f&\mapsto (a\mapsto z\wedge f(a))
        \end{align*}
    \end{itemize}
    Then there is an adjunction $$\Hom_{G\mathsf{Sp}}(X, \text{map}_*(A,Y))\cong\Hom_{G\mathsf{Sp}}(X\wedge A, Y)$$
    Replacing $A$ by a representation sphere $S^W$, these constructions give us \textit{suspensions} and \textit{loop spaces} respectively, which will be very relevant in the next talk.
\end{cons}

\begin{cons}[Shifts]
    For a $G$-representation $V$, we define the \textit{$V$-th shift} of an orthgonal ($G$)-spectrum $X$ to be $$sh^VX(U)\coloneqq X(U\oplus V)$$ with $O(U)$ acting through the homomorphism $-\oplus V:U\to U\oplus V$ and structure maps $$\sigma^{sh^VX}_{U,W}\coloneqq \sigma^X_{U, W\oplus V}$$
    Importantly, shifts commute with the smash and map constructions from before.
\end{cons}

The shift will be very useful later, for example in providing us with ways to change the indices in homotopy groups. We have given very brief overviews of the previous two constructions, but the one we will define now is important already in the definition of the homotopy groups of orthogonal $G$-spectra.

\begin{cons}[Induced map by an inclusion of representations]
    Let $\varphi:V\hookrightarrow W$ be a $G$-equivariant linear isometric embedding and $f:S^{V\oplus\bR^{n+k}}\to X(V\oplus\bR^n)$ a based continuous $G$-map, where $n\in\bN$ and $k\in\bZ$ such that $n+k\geq 0$. We define a map $$\varphi_*f:S^{W\oplus\bR^{n+k}}\to X(W\oplus\bR^n)$$ as the composite
    \[\begin{tikzcd}
        S^{W\oplus\bR^{n+k}} \arrow[r, "\cong"] \arrow[dd, "\varphi_*f"{description}] & S^{W-\varphi(V)}\wedge S^{V\oplus \bR^{n+k}} \arrow[d, "S^{W-\varphi(V)}\wedge f"] \\
        & S^{W-\varphi(V)}\wedge X(V\oplus \bR^{n+k}) \arrow[d, "\sigma_{W-\varphi(V), V\oplus \bR^n}"] \\
        X(W\oplus\bR^n) &  X((W-\varphi(V))\oplus V\oplus \bR^n) \arrow[l, "\cong"']
    \end{tikzcd}\] 
    where the horizontal maps are induced by the linear isometry 
    \begin{align*}
        (W-\varphi(V))\oplus V\oplus \bR^m &\xrightarrow{\cong} W\oplus \bR^m \\
        (w,v,x)&\mapsto (w+\varphi(v),x)
    \end{align*}
    for $m=n$ or $n+k$.
\end{cons}

\section{Equivariant Homotopy Groups}

Earlier we defined the poset $s(\mathcal{U}_G)$, ordered by inclusion. Knowing the definition of the ``usual'' stable homotopy groups of spectra, one might suspect that we now have to take some colimit over this poset. We will now describe the functor we will be taking colimit of, starting with $\pi^G_0$.

\begin{cons}
    Define a functor 
    \begin{align*}
        s(\mathcal{U}_G)&\to \mathsf{Set} \\
        V&\mapsto [S^V,X(V)]^G_*
    \end{align*} where $[S^V,X(V)]^G_*$ is the set of $G$-equivariant homotopy classes of based $G$-maps. On morphisms, we take an inclusion $V\xhookrightarrow{i} W$ to the map 
    \begin{align*}
        i_*: [S^V,X(V)]^G_*&\to [S^W,X(W)]^G_* \\
        [f] &\mapsto [i_*f]
    \end{align*}
\end{cons}

\begin{defin}
    The 0th \textbi{equivariant homotopy group} $\pi_0^G(X)$ is the colimit $$\pi_0^G(X)\coloneqq\colim_{V\in s(\mathcal{U}_G)}[S^V,X(V)]^G_*$$
\end{defin}

We now have to make this homotopy group actually be a \textit{group}. For this, we work with the $G$-equivariant homotopy classes to get bijections to another group. The following segment is taken directly from \cite[3.10]{GHT}.
\begin{cons}[Group structure on $\pi^G_0$]
    Let $V$ be a finite-dimensional $G$-subrepresentation of $\mathcal{U}_G$ with nontrivial fixed points, and let $v_0\in V$ be a $G$-fixed unit vector. Denote by $V^\perp$ the orthogonal complement of $v_0$ in $V$. Then the decomposition \begin{align*}
        \bR\oplus V^\perp&\xrightarrow{\cong} V\\
        (t,v) &\mapsto tv_0+v
    \end{align*} extends to one-point compactification and becomes a $G$-equivariant homeomorphism $S^1\wedge S^{V^\perp}\cong S^V$. Then using the smash-hom adjunction (now for based $G$-spaces) gives us a natural (in $X$) bijection $$[S^V, X(V)]^G_*\cong [S^1, \text{map}^G_*(S^{V^\perp}, X(V))]_*=\pi_1(\text{map}^G_*(S^{V^\perp}, X(V))$$
    We then give $[S^V, X(V)]^G_*$ a group structure by making this bijection a group isomorphism.
\end{cons}

\begin{prop}
    Let $V$ be such that the fixed point space $V^G$ has dimension at least 2. Then
    \begin{enumerate}
        \item The group structure on $[S^V, X(V)]^G_*$ is commutative and independent of the choice of $G$-fixed unit vector.
        \item  If $W$ is another finite-dimensional $G$-subrepresentation of $\mathcal{U}_G$ containing $V$, then the map $$i_*:[S^V, X(V)]^G_*\to [S^W, X(W)]^G_*$$ is a group homomorphism.
    \end{enumerate}
\end{prop}

Note that for some $V\in s(\mathcal{U}_G)$, we can embed $V$ into a higher-dimensional representation in $s(\mathcal{U}_G)$ in which the action leaves everything not in $V$ fixed. In other words, the $G$-subrepresentations that have at least 2-dimensional fixed point spaces are cofinal in $s(\mathcal{U}_G)$. Now, by part (2.) we get that the group structure carries over to the colimit $\pi^G_0(X)$, and (1.) tells us that this group structure is abelian. 

We now use the same idea to introduce $\bZ$-graded equivariant homotopy groups, which also come with an abelian group structure. 

\begin{defin}
    Let $X$ be an orthogonal $G$-spectrum and $k\in\bN$. We set
    \begin{itemize}
        \item $\pi^G_k(X)\coloneqq\colim_{V\in s(\mathcal{U}_G)}[S^{V\oplus \bR^k},X(V)]^G_*$
        \item $\pi^G_{-k}(X)\coloneqq\colim_{V\in s(\mathcal{U}_G)}[S^{V},X(V\oplus \bR^k)]^G_*$
    \end{itemize}
\end{defin}

\begin{defin}
    A morphism $f:X\to Y$ of orthogonal $G$-spectra is a $\underline{\pi}_*$-\textbi{isomorphism} if the induced map $\pi^H_k(f):\pi^H_k(X)\to\pi^H_k(Y)$ is an isomorphism for all $k\in\bZ$ and all closed subgroups $H\leq G$.
\end{defin}

There is one final bit of subtlety we need to deal with: We need to talk about how $G$-maps from representation spheres (with reprect to \textit{any} finite-dimensional $G$-representation) to spaces in our spectrum represent well-defined classes in the equivariant homotopy groups.

\begin{cons}
    Let $f:S^{V\oplus \bR^{n+k}}\to X(V\oplus \bR^n)$ be a $G$-equivariant map, where $k\in \bZ$, $n\in\bN$, $n+k\geq 0$ and $V$ is a finite-dimensional $G$-representation. \\ \\
    \begin{minipage}{.475\textwidth}
        For $k\geq 0$, choose a $G$-equivariant linear isometry $$j:V\oplus\bR^n\to \Tilde{V}\in s(\mathcal{U}_G)$$ Define $\langle f\rangle\in\pi^G_k(X)$ to be the class represented by the composite 
        \[\begin{tikzcd}
            S^{\Tilde{V}\oplus\bR^k} \arrow[r, "(S^{j\oplus \bR^k})^{-1}", "\cong"'] \arrow[d] & S^{V\oplus\bR^{n+k}} \arrow[d, "f"] \\ X(\Tilde{V}) &  X(V\oplus\bR^n) \arrow[l, "X(j)", "\cong"']
        \end{tikzcd}\]
    \end{minipage}  
    \begin{minipage}{.05\textwidth} \ \end{minipage}
    \begin{minipage}{.475\textwidth}
        For $k\leq 0$, choose a $G$-equivariant linear isometry $$j:V\oplus\bR^{n+k}\to \Tilde{V}\in s(\mathcal{U}_G)$$ Define $\langle f\rangle\in\pi^G_k(X)$ to be the class represented by the composite
        \[\begin{tikzcd}
            S^{\Tilde{V}} \arrow[r, "(S^{j})^{-1}", "\cong"'] \arrow[d] & S^{V\oplus\bR^{n+k}} \arrow[d, "f"] \\ X(\Tilde{V}) &  X(V\oplus\bR^{-k}) \arrow[l, "X(j)", "\cong"']
        \end{tikzcd}\]
    \end{minipage}
\end{cons}

Note that there is a choice of isometry in this construction, so we make sure that this does not change thins, and also that taking embeddings again leaves the classes unchanged.

\begin{prop}{\normalfont \cite[3.1.14]{GHT}}
    Let $G$ be a compact Lie group and $X$ an orthogonal $G$-spectrum. Let $V$ be a $G$-representation and $f:S^{V\oplus\bR^{n+k}}\to X(V\oplus\bR^n)$ a based continuous $G$-map, where $n\in\bN$ and $k\in\bZ$ such that $n+k\geq 0$. Then:
    \begin{enumerate}
        \item The class $\langle f\rangle\in\pi^G_k(X)$ is independent of the choice of linear isometry onto a subrepresentation of $\mathcal{U}_G$.
        \item For every $G$-equivariant isometric embedding $\varphi:V\hookrightarrow W$, the relation $\langle \varphi_*f\rangle = \langle f\rangle$ holds in $\pi^G_k(X)$.
    \end{enumerate}
\end{prop}

\begin{thebibliography}{}
    \bibitem[Sch18]{GHT}{S. Schwede, \href{https://www.math.uni-bonn.de/people/schwede/global.pdf}{Global homotopy theory}. New Mathematical Monographs 34. Cambridge University Press, 2018.}
    \bibitem[Sch23]{Sch23}{S. Schwede, \href{https://www.math.uni-bonn.de/people/schwede/equivariant.pdf}{Lecture notes on equivariant stable homotopy theory}.}
    \bibitem[AGP]{AGP}{M. Aguilar, S. Gitler, C. Prieto, \href{https://link.springer.com/book/10.1007/b97586}{Algebraic topology from a homotopical viewpoint}. Springer, 2002.}
    \bibitem[St]{St}{ M. Stolz, \href{https://people.math.rochester.edu/faculty/doug/otherpapers/mstolz.pdf}{Equivariant structure on smash powers of commutative ring spectra}. PhD thesis, University of Bergen, 2011.}
    \bibitem[MM]{MM}{M. A. Mandell, J. P. May, \href{https://www.math.uchicago.edu/~may/PAPERS/MMMFinal.pdf}{Equivariant orthogonal spectra and S-modules}. Mem. Amer. Math. Soc. 159 (2002), no. 755}
    \bibitem[HHR]{HHR}{M. Hill, M. Hopkins, D. Ravenel, \href{https://annals.math.princeton.edu/wp-content/uploads/annals-v184-n1-p01-p.pdf}{On the nonexistence of elements of Kervaire invariant one}. Ann. of Math. 184 (2016)}
\end{thebibliography}

\end{document}

